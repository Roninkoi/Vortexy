\documentclass[12pt]{article}
\usepackage[dvips]{graphicx}
\usepackage{color}
\usepackage[finnish, english]{babel}
\usepackage[utf8]{inputenc}
\usepackage{wrapfig}
\usepackage{caption}
\usepackage{amsmath}
\usepackage{amsfonts}
\usepackage{amsmath}
\usepackage{fancyhdr}
\usepackage{titling}
\usepackage[top=52pt, bottom=2cm, left=2cm, right=2cm]{geometry}
\usepackage{float}
\usepackage{hyperref}
\usepackage{authblk}
\usepackage{comment}

\pagestyle{fancy}

\title {
  Vortexy fluid dynamics simulator
}

\date{\today}

\def \firstauth{
  Roni Koitermaa
}

\def \email{roninkoi@iki.fi}

\author[1] {
  \firstauth\thanks{\href{mailto: \email}{\email}}
}

\fancyhf{}
\setlength{\headheight}{15pt}
\lhead{\thetitle}
\rhead{\firstauth}
\cfoot{\thepage}

\renewcommand\maketitlehooka{\vspace{0.2\textheight}}
% \renewcommand\maketitlehookd{\vfill}

% CUSTOM COMMANDS
% characters
% \newcommand{\tmu}{\ensuremath{\mu}}
% math
\newcommand{\eint}[2]{\mathrel{ \substack{#2\\ \bigg /\\ #1}}}
\newcommand{\D}{\text{d}}
\newcommand{\BD}{\text{D}}
\newcommand{\I}{\text{i}}
\newcommand{\mln}{\overline{\ln}}
\newcommand{\Ln}{\text{Ln}}
\newcommand{\Arg}{\text{Arg}}
\newcommand{\Res}{\text{Res}}
\newcommand{\Ind}{\text{Ind}}
\newcommand{\lb}{\left(}
  \newcommand{\rb}{\right)}
\newcommand{\Imag}{\text{Im} \ }
\newcommand{\Real}{\text{Re} \ }
\newcommand{\ft}{\mathcal{F}}
\newcommand{\lt}{\mathcal{L}}
% units
\newcommand{\us}[1]{ \ \text{#1}}
\newcommand{\un}[1]{ \text{#1}}
\newcommand{\uf}[2]{ \ \frac{\text{#1}}{\text{#2}}} % fraction of units
\newcommand{\mic}{$\upmu$}
\newcommand{\degc}{ \ ^\circ\text{C}}
\newcommand{\degs}{^\circ}
\newcommand{\es}{\text{e}}
\newcommand{\tp}[1]{ \times 10^{#1}}
\newcommand{\vb}[1]{\text{\textbf{#1}}}

\newcommand{\x}{\text}

\begin{document}

\setlength{\belowcaptionskip}{10pt}

\selectlanguage{english}

\normalsize

\begin{titlingpage}
  \maketitle

  % \begin{abstract}
  % \end{abstract}
  \begin{center}
    Software documentation
  \end{center}
\end{titlingpage}

\newpage

\tableofcontents

\newpage

\section{Introduction}

{\bf Vortexy} is a computational fluid dynamics (CFD) simulation package. It is written in C and uses the finite volume method with the SIMPLE algorithm to calculate flow of incompressible fluids, namely liquids.

The simulator is based on irregular tetrahedral meshes. These meshes can be computed from surfaces using the program Tetgen. The simulator takes a configuration file as input that contains paths to the simulation mesh and boundary conditions in addition to other settings. The state of the system is periodically written to an output file specified in the config. Included is also a renderer that uses OpenGL to visualize results.

\section{Background}

\subsection{Navier-Stokes equations}

\noindent

The Navier-Stokes equations form the basis for all of fluid dynamics. The momentum equation is typically written as

\begin{equation}
  \frac{\partial \vb u}{\partial t} + (\vb u \cdot \nabla) \vb u = -\frac{1}{\rho} \nabla p + \nu \nabla^2 \vb u + g,
\end{equation}

where $\vb u = (u, v, w)$ is velocity in [m/s], t time in [s], $\rho$ density in [kg/m$^3$], $p$ pressure in [Pa], $\nu=\frac{\mu}{\rho}$ kinematic viscosity in [m$^2$/s], $g$ gravity in [m/s].

The continuity equation must be satisfied for incompressible fluids that have no sinks of sources

\begin{equation}
  \nabla \cdot \vb u = 0,
\end{equation}

where $\vb u = (u, v, w)$ is velocity in [m/s].

\subsection{Turbulence}

\noindent

A simple way of predicting onset of turbulence is the Reynolds number:

$$
\text{Re} = \frac{\mu u L}{\rho} = \frac{u L}{\nu}
$$

Turbulence models in simulations include RANS (Reynolds Averaged), LES (Large Eddy) and DNS (Direct).

\subsection{Finite volume method}

The \textit{finite volume method} (FVM) is based on a simulation mesh with volume elements. This enables evaluation of partial differential equations (PDEs) prevalent in physics. The divergence theorem allows us to convert volume integrals to surface integrals

$$
\int_V \nabla \cdot \vb F \ \D V = \oint_S \vb F \cdot \D \vb S,
$$

\noindent
so volume terms can be computed from fluxes at element faces.

\subsection{Discretization}

\noindent

The momentum equation is written in a form conducive for discretization:
$$
\frac{\partial \vb u}{\partial t} + \nabla \cdot (\vb u \otimes \vb u) = - \frac{\nabla p}{\rho} + \frac{\mu}{\rho} \nabla \cdot (\nabla u) + \nabla \cdot (\nabla u)^T + f_b
$$
$$
\vb{transient} + \vb{convective} =  \vb{diffusion} + \vb{source}
$$

Continuity equation:
$$
\nabla \cdot \vb u = 0
$$

This saddle-point problem can be represented in matrix form as:

$$
A \vb u = \begin{pmatrix}
  F & B^T \\
  B & 0
\end{pmatrix}
\begin{pmatrix}
  \vb u \\
  p
\end{pmatrix} =
\begin{pmatrix}
  \vb f_b \\
  0
\end{pmatrix}
$$
  In practice, the velocity and pressure fields are calculated using 4 matrices (vx, vy, vz, p).

Discretization for the one-dimensional momentum equation \cite{mou}:

\begin{equation}
  a_v u_v + \sum_{f \in f_{nb}} a_f u_f = b_v,
\end{equation}

which can be represented in matrix form as:

$$
\begin{pmatrix}
  a_{00} & a_{01} & \dots & a_{0n} \\
  \vdots & \ddots &  & \vdots \\
  \vdots &  & \ddots & \vdots \\
  a_{n0} & \dots & a_{n (n - 1)} & a_{0n}
\end{pmatrix}
\begin{pmatrix}
  u_0 \\
  \vdots \\
  u_n
\end{pmatrix} =
\begin{pmatrix}
  b_{0} \\
  \vdots \\
  b_n
  \end{pmatrix},
  $$

  where $u_0 \dots u_n$ represents volume element velocities. In the pressure calculation these are referred to as $u^*$, i.e. momentum conserving. The coefficients $a$ and $b$ are calculated for each volume element and assembled into a matrix. Components X, Y and Z are calculated one after another.\\

  Face fluxes:
  \begin{align}
    \phi_f = \max(\dot m_f, 0) + \mu \frac{E_f}{d_{vf}} \\
    \Phi_f = -\max(-\dot m_f, 0) - \mu \frac{E_f}{d_{vf}} \\
    \vec \psi_f = - \mu (\nabla \vb u_f) \cdot \vb T_f + \dot m_f (\vb u_f^{\text{hr}} - \vb u_f^{\text{uw}}),
  \end{align}
  where $\vb S_f = A_f \hat n_f = \vb E_f + \vb T_f$ and $\dot m_f = \rho \vb u_f \cdot \vb S_f$. High-resolution model velocity $\vb u_f^{\text{hr}} = \vb u_f$ and upwind velocity $\vb u_f^{\text{uw}}$.\\

  Volume fluxes:
    \begin{align}
      \phi_v = \frac{\rho V}{\Delta t} \\
      \vec \psi_v = \frac{\rho V}{\Delta t} \vb u_v - \vb f_b V
    \end{align}

    Coefficients:
  \begin{align}
    a_v = \phi_v + \sum_{f \in f_{nb}} \phi_f = \frac{\rho V}{\Delta t} + \sum_{f \in f_{nb}} \lb \max(\dot m_f, 0) + \mu \frac{E_f}{d_{vf}} \rb \\
    a_f = \Phi_f = -\max(-\dot m_f, 0) - \mu \frac{E_f}{d_{vf}}
  \end{align}
  \begin{equation}
    \vb b_v = -\vec \psi_v - \sum_{f \in f_{nb}} \vec \psi_f + \sum_{f \in f_{nb}} \lb \mu (\nabla \vb u_f)^T \cdot \vb S_f \rb - V \nabla p_v
    \end{equation}
  $$
    = - \frac{\rho V}{\Delta t} \vb u_v - \vb f_b V - \sum_{f \in f_{nb}} \lb - \mu (\nabla \vb u_f) \cdot \vb T_f + \dot m_f (\vb u_f - \vb u_f^{\text{uw}}) \rb + \sum_{f \in f_{nb}} \lb \mu (\nabla \vb u_f)^T \cdot \vb S_f \rb - V \nabla p_v
$$

Pressure equation:

\begin{equation}
  u_v^* + \sum_{f \in f_{nb}} \frac{a_f}{f_v} \vb u_f^* = -\frac{V}{a_v} \nabla p_v + \frac{b_v + V \nabla p_v}{a_v}
  \end{equation}

  Similarly, the pressure correction equation can be discretized \cite{mou}:

  \begin{equation}
    a_v p_v' + \sum_{f \in f_{nb}} a_f p_f' = b_v
  \end{equation}

  With coefficients:

  \begin{align}
    a_f = -\rho \frac{E_f}{d_{vf}} \\
    a_v = \sum_{f \in f_{nb}} \rho \frac{E_f}{d_{vf}}\\
    b_v = - \sum_{f \in f_{nb}} \dot m_f^* = - \sum_{f \in f_{nb}} \rho \vb u_f^* \cdot \vb S_f
    \end{align}

    All of these are $n \times n$ diagonally dominant matrices which can be solved using a numerical matrix solver, such as Gauss-Seidel. The off-diagonal coefficients represent neighbouring elements, so they will be mostly zero.

\subsection{Gauss-Seidel method}

\subsection{SIMPLE algorithm}

\begin{enumerate}
\item Set boundary conditions, set $u$ and $p$
\item Compute gradients $\nabla u$ and $\nabla p$
\item Compute mass fluxes $j_m$, flow rate $\dot m = j_m \cdot A$
\item Solve \textit{momentum equation} using velocity guess $u^0$
\item Solve \textit{pressure correction equation} to get $p'$
\item Correct pressure $p = p + p'$ and velocity
\item Increment time $t^{n+1} = t^n + \Delta t$
\item Repeat
\end{enumerate}

\subsection{Boundary conditions}

\noindent

No-slip wall\\

Slip wall\\

Inlet\\

Outlet\\

\section{Implementation}

\subsection{Simulation mesh}

Triangular/tetrahedral, closed, connected, Delaunay

\subsection{Solver}

Gauss-Seidel

\subsection{Rendering}

OpenGL

\section{Software}

\subsection{Compilation}

Deps: glibc, (OpenGL, GLEW, GLFW)

\begin{verbatim}
cmake .
make
\end{verbatim}

\subsection{Configuration}

Files: sim.cfg, data/fluid.cfg

\subsection{Examples}

\subsubsection{Lid-driven cavity}

\subsubsection{Pipe flow}

\subsection{Problems}

Checkerboard problem

\addcontentsline{toc}{section}{References}
\begin{thebibliography}{9}
\bibitem{mou} Moukalled, F., Mangani, L. \& Darwish M. (2016). The Finite Volume Method in Computational Fluid Dynamics: An Advanced Introduction with OpenFOAM and Matlab. Fluid Mechanics and Its Applications Volume 113. Springer International Publishing, Cham. \url{https://doi.org/10.1007/978-3-319-16874-6}
  
\bibitem{qs} \url{https://quickersim.com/tutorial/tutorial-2-numerics-simple-scheme/}
\bibitem{ofs} \url{https://www.openfoam.com/documentation/guides/latest/doc/guide-applications-solvers-simple.html}
\bibitem{cws} \url{https://www.cfd-online.com/Wiki/SIMPLE_algorithm}

\end{thebibliography}

\end{document}

% Local Variables:
% coding: utf-8-unix
% TeX-engine: luatex
% End:
